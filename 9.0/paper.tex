\documentclass{ansarticle-preprint}
%\usepackage{ucs}
\usepackage[utf8]{inputenc}
\usepackage{amsmath}
%\usepackage{cite}
\usepackage{anslistings}
\usepackage{multicol}

\usepackage{pgfplots}
\usepackage{pgfplotstable}

\usepackage{fontenc}
\usepackage{graphicx}
\usepackage{xspace}

\pgfplotsset{compat=1.9}

\newcommand{\specialword}[1]{\texttt{#1}}
\newcommand{\dealii}{{\specialword{deal.II}}\xspace}
\newcommand{\pfrst}{{\specialword{p4est}}\xspace}
\newcommand{\trilinos}{{\specialword{Trilinos}}\xspace}
\newcommand{\aspect}{\specialword{Aspect}\xspace}
\newcommand{\petsc}{\specialword{PETSc}\xspace}
\newcommand{\cmake}{{\specialword{CMake}}\xspace}
\newcommand{\autoconf}{{\specialword{autoconf}}\xspace}

%
% Author list -- please add yourself in both places below (in
%                alphabetical order) if you think that your
%                contributions to the last release warrant this
%

\hypersetup{
  pdfauthor={
    Giovanni Alzetta,
    Daniel Arndt,
    Wolfgang Bangerth,
    Vishal Boddu,
    Benjamin Brands,
    Denis Davydov,
    Rene Gassm\"{o}ller,
    Timo Heister,
    Luca Heltai,
    Katharina Kormann,
    Martin Kronbichler,
    Matthias Maier,
    Jean-Paul Pelteret,
    Bruno Turcksin,
    David Wells
  },
  pdftitle={The deal.II Library, Version 9.0, 2018},
}

\title{The \dealii Library, Version 9.0}

\author[1]{Giovanni Alzetta}
\affil[1]{SISSA,
  International School for Advanced Studies,
  Via Bonomea 265,
  34136, Trieste, Italy.
{\texttt{\{galzetta,luca.heltai\}@sissa.it}}}

\author[2]{Daniel Arndt}
\affil[2]{Interdisciplinary Center for Scientific Computing,
  Heidelberg University,
  Im Neuenheimer Feld 205,
  69120 Heidelberg, Germany.
  {\texttt{daniel.arndt@iwr.uni-heidelberg.de}}}

\author[3]{Wolfgang Bangerth}
\affil[3]{Department of Mathematics, Colorado State University, Fort
  Collins, CO 80523-1874, USA.
    {\texttt{bangerth@colostate.edu}}}

\author[4]{Vishal Boddu}
\affil[4]{Chair of Applied Mechanics,
  Friedrich-Alexander-Universit\"{a}t Erlangen-N\"{u}rnberg,
  Egerlandstr.\ 5,
  91058 Erlangen, Germany.
  {\texttt{\{vishal.boddu,benjamin.brands,denis.davydov,jean-paul.pelteret\}@fau.de}}}

\author[4]{Benjamin Brands}

\author[4]{Denis Davydov}

\author[5]{Rene Gassm\"{o}ller}
\affil[5]{Department of Earth and Planetary Sciences,
  University of California Davis,
  One Shields Avenue,
  CA-95616 Davis, USA.
  {\texttt{rgassmoeller@ucdavis.edu}}}

\author[6]{Timo Heister}
\affil[6]{Mathematical Sciences,
  O-110 Martin Hall,
  Clemson University,
  Clemson, SC 29634, USA.
  {\texttt{heister@clemson.edu}}}
%
\author[1]{Luca Heltai}

\author[7]{Katharina Kormann}
\affil[7]{Max Planck Institute for Plasma Physics,
Boltzmannstr.~2, 85748 Garching, Germany.
{\texttt{katharina.kormann@ipp.mpg.de}}}

\author[8]{Martin Kronbichler}
\affil[8]{Institute for Computational Mechanics,
  Technical University of Munich,
  Boltzmannstr.~15, 85748 Garching, Germany.
  {\texttt{kronbichler@lnm.mw.tum.de}}}

\author[9]{Matthias Maier}
\affil[9]{School of Mathematics,
  University of Minnesota,
  127 Vincent Hall, 206 Church Street SE,
  Minneapolis, MN 55455, USA.
  {\texttt{msmaier@umn.edu}}}

\author[4]{Jean-Paul Pelteret}

 \author[10]{Bruno Turcksin\footnote{
   This manuscript has been authored by UT-Battelle, LLC under Contract No.
   DE-AC05-00OR22725 with the U.S. Department of Energy. The United States
   Government retains and the publisher, by accepting the article for
   publication, acknowledges that the United States Government retains a
   non-exclusive, paid-up, irrevocable, worldwide license to publish or reproduce
   the published form of this manuscript, or allow others to do so, for United
   States Government purposes. The Department of Energy will provide public
   access to these results of federally sponsored research in accordance with the
   DOE Public Access Plan (http://energy.gov/downloads/doe-public-access-plan).}}
 \affil[10]{Computational Engineering and Energy Sciences Group,
   Computional Sciences and Engineering Division,
   Oak Ridge National Laboratory, 1 Bethel Valley Rd.,
   TN 37831, USA.
   {\texttt{turcksinbr@ornl.gov}}}

\author[11]{David Wells}
\affil[11]{Department of Mathematical Sciences, Rensselaer Polytechnic
Institute, Troy, NY 12180, USA.
  {\texttt{wellsd2@rpi.edu}}}

\renewcommand{\labelitemi}{--}


\begin{document}
\maketitle

\begin{abstract}
  This paper provides an overview of the new features of the finite element
  library \dealii version 9.0.
\end{abstract}



%%%%%%%%%%%%%%%%%%%%%%%%%%%%%%%%%%%%%%%%%%%%%%%%%%%%%%%%%%%%%%%%%%%%%%%%%%%%%%%%
%%%%%%%%%%%%%%%%%%%%%%%%%%%%%%%%%%%%%%%%%%%%%%%%%%%%%%%%%%%%%%%%%%%%%%%%%%%%%%%%
%%%%%%%%%%%%%%%%%%%%%%%%%%%%%%%%%%%%%%%%%%%%%%%%%%%%%%%%%%%%%%%%%%%%%%%%%%%%%%%%
\section{Overview}

\dealii version 9.0.0 was released April X, 2018.
\marginpar{Update the date}
This paper provides an
overview of the new features of this major release and serves as a citable
reference for the \dealii software library version 9.0. \dealii is an
object-oriented finite element library used around the world in the
development of finite element solvers. It is available for free under the
GNU Lesser General Public License (LGPL) from the \dealii homepage at
\url{http://www.dealii.org/}.

The major changes of this release are:
\begin{itemize}
\item Improved support for curved geometries;
\item Support for particle-in-cell methods;
\item Dedicated support for automatic differentiation;
\item Interfaces to more external libraries and programs;
\item C++11 is now both required and used;
\item Support for GPU computations;
\item Support for face integrals and significant improvements of the matrix-free framework;
\end{itemize}
These will all be discussed in more detail in the
following section. In addition, this release contains the following changes:
\begin{itemize}
\item \dealii has made extensive use of both the Clang-Tidy \cite{clang-tidy}
  and Coverity Scan \cite{coverity} static analysis tools for detecting bugs
  and other issues in the code. For example, around 260 issues were detected and
  fixed using the latter tool.

\item
  \texttt{LinearOperator}, a flexible template class that implements the action of a
  linear operator (see \cite{MaierBardelloniHeltai-2016-b}), now supports
  computations with Trilinos, Schur complements, and linear constraints. This
  class is, as of this release, the official replacement for about half a dozen
  similar (but less general) classes, such as \texttt{FilteredMatrix},
  \texttt{IterativeInverse}, and \texttt{PointerMatrix}.

\item New non-standard quadrature rules:
A number of non-standard, special-purpose quadrature rules have been
implemented. Among these are ones for
(i) truncating standard formulas to simplical domains (\texttt{QSimplex});
(ii) singular transformations of the unit cell to the unit simplex
    (\texttt{QDuffy});
(iii) composition of simplical quadrature rules to a combined rule on the
    unit cell (\texttt{QSplit});
and (iv) transformation of the unit square to polar coordinates
    (\texttt{QTrianglePolar}).
These quadrature rules greatly help when integrating singular
functions or on singular domains. They
are mainly used in Boundary Element Methods.


\item
  Support for complex-valued vectors at the same level as real-valued
  vectors.

\item A new python tutorial program tutorial-1; as well as
  updates to step-37. In addition, the separate code
  gallery of \dealii has gained a number of new entries.

\item Improved support for user-defined run-time parameters: a new \texttt{ParameterAcceptor}
  class has been added to the library. The class is intended to be used as a
  base for any class that wants to handle parameters using the
  \texttt{ParameterHandler} class. If you derive all your classes from
  \texttt{ParameterAcceptor}, and declare your parameters either with
  \texttt{parse/declare\_parameters} methods or via the
  \texttt{ParameterAcceptor::add\_parameter} method, then both the
  declaration and the
  parsing of parameter files are automatically managed by
  \texttt{ParameterAcceptor::initialize}, greatly simplifying
  dealing with parameters in user codes.

\item New caching mechanism for expensive grid computations: we introduced a new
  class \texttt{GridTools::Cache} that caches computationally intensive
  information about a \texttt{Triangulation}.

This class allows the user to query some of the data structures constructed
using functions in the \texttt{GridTools} namespace. This data is then
computed only once, and
then cached inside this class for faster access whenever the triangulation has
not changed. The cache is marked for update by the \texttt{Triangulation} itself
using signals.

Some of the methods in \texttt{GridTools} already use this cache to
speed up repeated calls to the same expensive methods.
\item A new \texttt{MeshWorker::mesh\_loop} function has been added that
  performs the same tasks of the \texttt{MeshWorker::loop} function without
  forcing the users to adhere to a specific interface.

\end{itemize}
Beyond these changes, the changelog lists more than 330 other features and bugfixes.




%%%%%%%%%%%%%%%%%%%%%%%%%%%%%%%%%%%%%%%%%%%%%%%%%%%%%%%%%%%%%%%%%%%%%%%%%%%%%%%%
%%%%%%%%%%%%%%%%%%%%%%%%%%%%%%%%%%%%%%%%%%%%%%%%%%%%%%%%%%%%%%%%%%%%%%%%%%%%%%%%
%%%%%%%%%%%%%%%%%%%%%%%%%%%%%%%%%%%%%%%%%%%%%%%%%%%%%%%%%%%%%%%%%%%%%%%%%%%%%%%%
\section{Significant changes to the library}

This release of \dealii contains a number of large and significant changes
that will be discussed in the following sections. It of course also contains a
vast number of smaller changes and added functionality; the details of these
can be found
\href{https://www.dealii.org/developer/doxygen/deal.II/changes_between_8_5_and_9_0.html}{
in the file that lists all changes for this release}, see \cite{changes90}.
(The file is also linked to from the web site of each release as well as
the release announcement.)

%%%%%%%%%%%%%%%%%%%%%%%%%%%%%%%%%%%%%%%%%%%%%%%%%%%%%%%%%%%%%%%%%%%%%%%%%%%%%%%%
\subsection{Improved support for curved geometries}
\label{sec:manifolds}

\marginpar{All: please edit as appropriate}

\dealii{} has had the ability to attach \textit{manifold descriptions}
to all parts of a geometry for several releases already. These
descriptions are used when considering where to place new vertices
upon mesh refinement, in determining the mapping from reference cell
to real cell, and in a number of other contexts. However, for historical
reasons, manifold descripts have used some of the same code paths also
used for boundary indicators typically used to identify which parts of
the boundary correspond to what boundary conditions.

This connection has been severed in the current release: Boundary
indicators and manifold descriptions are now entirely separated. Furthermore,
all functions that allowed to query for old-style \texttt{Boundary} objects (deprecated
since version 8.5 of the \dealii library) have been removed, and have been
replaced by the equivalent methods that query \texttt{Manifold} classes.

There are also numerous improvements to the available manifold descriptions.
First, the manifold smoothing algorithms applied in the \texttt{Triangulation} class and
\texttt{MappingQGeneric} have been changed from the old Laplace-style smoothing to a
transfinite interpolation that linearly blends between the descriptions on the
faces around a cell. The old transformation introduced boundary layers inside
cells that prevented convergence rates from exceeding \(3.5\) in the global
\(L^2\) errors for typical settings. This change also considerably improves mesh
quality in situations where curved descriptions are only applied to the boundary
rather than the whole volume.

\marginpar{This paragraph seems to duplicate the previous. Can we
  merge the information?}
Secondly, a new manifold class
\texttt{TransfiniteInterpolationManifold} has been added that implements an
interpolation from a curved boundary description to the interior.
This class enables high-order convergence rates of better than ${\cal O}(h^3)$
for situations where a curved manifold can only be prescribed
to the boundary but not in a whole volume.

Finally,
every function in the \texttt{GridGenerator} namespace now attaches a default manifold to
the curved parts of the domain described by the generated mesh, and sets reasonable defaults for manifold
indicators both in the domain and on the boundary.



%%%%%%%%%%%%%%%%%%%%%%%%%%%%%%%%%%%%%%%%%%%%%%%%%%%%%%%%%%%%%%%%%%%%%%%%%%%%%%%%
\subsection{Support for particle-in-cell methods}

\marginpar{Rene: Please edit as appropriate}
While \dealii is a package intended to solve problems with the
finite element method -- i.e., using continuous or discontinuous
\textit{fields} --, it is often convenient in fluid dynamics problems to
couple the continuum description of phenomena with particles. These
particles, advected along with the numerical approximation of the flow
field, are then either used to visualize properties of the flow, or to
advect material properties such as the viscosity of inhomogeneous
mixtures of fluids. If each particle is associated with the cells
of a mesh, these methods are often referred to as particle-in-cell (PIC).

\dealii now has a dedicated particles module. The module provides a base class
\texttt{Particle} that represents a particle with position, an ID number and
a variable number of properties. They are jointly represented by a \texttt{ParticleHandler}
class that manages the storage and handling of all particles. In
parallel simulations, this class also
distributes the particles among the subdomains of the parallel process
and supports efficient data transfer during mesh refinement and
checkpoint/restart phases.

A much more detailed view of the underlying algorithms can be found in
\cite{GLHPB18}. A longer report is at \cite{GHPB16}. The
implementation here originated in the \aspect code, see \cite{KHB12,HDGB17}.


%%%%%%%%%%%%%%%%%%%%%%%%%%%%%%%%%%%%%%%%%%%%%%%%%%%%%%%%%%%%%%%%%%%%%%%%%%%%%%%%
\subsection{Dedicated support for automatic differentiation}

Automatic differentiation (AD) is often used to automatically derive
residuals and their linearization from a stored energy functional, and to
derive Jacobian matrices from residual vectors for simulations that use
complicated material models. Examples of its application can be found widely
within nonlinear solid mechanics, coupled multiphysics problems, as well as
for nonlinear viscosity models in fluid flow.

\dealii has had a tutorial program (step-33) since 2007 that
demonstrates this technique based on the Trilinos Sacado \cite{Bartlett2006a} package, but
the functionality was not available pervasively throughout
deal.II. 
This has changed with release 9.0 where support is given for differentiation 
using a selection of ``white-listed'' libraries (namely ADOL-C \citep{Griewank1996a} 
and Sacado) and a subset of their supported number types. Currently,
we offer support for the following cases:
\begin{itemize}
\item ADOL-C taped (n-differentiable),
\item ADOL-C tapeless (once differentiable),
\item Sacado dynamic forward (once differentiable),
\item Sacado reverse (once differentiable),
\item Sacado nested dynamic forward (twice differentiable), and
\item Sacado nested reverse and dynamic forward (twice differentiable).
\end{itemize}
In practice, this support means that these ADOL-C and Sacado data
types can be used in the \texttt{FEValues}, \texttt{FEValuesViews},
\texttt{Tensor}, \texttt{SymmetricTensor},
and related classes that are generally used to assemble linear systems
and right hand sides.
Given the updated capabilities of the library, there is now a dedicated module 
that presents the AD compatibility and capabilities of the \dealii libraries. 
Furthermore, the use of Sacado is now  demonstrated in a much more simplified 
and transparent manner in an modernized version of an existing ``code gallery'' 
example \cite{Pelteret2016a}. 

To date it remains necessary for the user  to manage the initialization of AD 
independent variables and the resultant calls  to the AD dependent variables 
in order to initiate the computation of derivatives.
In the next release we expect to provide a unified interface to these
AD libraries, which will hide these library-dependent implementational details 
and facilitate switching between the supported libraries and AD number types
based on the user's requirements.


%%%%%%%%%%%%%%%%%%%%%%%%%%%%%%%%%%%%%%%%%%%%%%%%%%%%%%%%%%%%%%%%%%%%%%%%%%%%%%%%
\subsection{New interfaces to external libraries and programs}

\dealii has always tried to leverage high-quality implementations of
algorithms available through other open source software, rather than
re-implementing their functionality. (A list of interfaces to other
packages is given in Section~\ref{sec:cite}.) As part of the current
release, we have written several new interfaces as discussed in the following.

\paragraph*{Assimp, the Open Asset Import Library.}
  Assimp \cite{assimp} can be used to read about 40 different 3D
  graphics formats. A subset of these formats can be now be read from
  within \dealii to generate two-dimensional meshes, possibly embedded in
  a three-dimensional space.

% \paragraph*{Gmsh.}
%   \marginpar{Do we actually provide any functionality with the GMSH
%   executable yet? If so, complete this part}
%   Gmsh \cite{geuzaine2009gmsh}
% The only thing we currently do is to define DEAL_II_GMSH_EXECUTABLE_PATH

\paragraph*{nanoflann, a library for building and querying
  $k$-d trees of datasets.} Operations such as finding the vertex or cell
  closest to a given evaluation point occur frequently in many applications
  that use unstructured meshes. While the naive algorithm is linear in the
  number of vertices or cells, many such operations can be made significantly
  faster by building a $k$-d tree data structure that recursively subdivides
  a $k$ dimensional space. The nanoflann library \cite{nanoflann} provides
  such a data structure and allows querying it, either for closest points
  (e.g., when finding the closest vertex) or for searching the points that
  fall within a radius of a target point. This functionality is now available
  via \dealii interfaces.

\paragraph*{ROL, a Rapid Optimization Library.}
  ROL \cite{ridzal2014rapid} is a package for large-scale optimization.
  \dealii can now use the state-of-the-art algorithms in ROL to solve
  unconstrained and constrained optimization problems as well as optimization
  problems under uncertainty.
  \dealii provides an interface to ROL's (abstract) vector class using the
  adapter software pattern.
  Through such an interface any vector class in \dealii following certain
  interface requirements can be used to define a ROL objective function.

\paragraph*{ScaLAPACK, a parallel dense linear algebra library for distributed memory machines.}
  ScaLAPACK \cite{slug} provides block-cyclic matrix distribution over 2D
  process grids. The functionality and interface of our wrappers is similar
  to the LAPACK \cite{lapack} wrappers for serial dense linear algebra, namely
  matrix-matrix multiplication, Cholesky and LU factorizations,
  eigensolvers, SVD, least squares, pseudoinverses and save/load operations using
  either serial or parallel HDF5 \cite{hdf5}. All of this functionality is available
  even in cases where the number of MPI processes does not match the numbers of
  processes in the 2D process grid used to distribute a matrix.

  As part of this effort, we have also improved LAPACK support: there
  are now methods to perform rank-1 updates/downdates, Cholesky
  factorizations, to compute the trace and determinant, as well as
  estimate the reciprocal condition number. We also now support
  configuration with 64-bit BLAS.

\paragraph*{SUNDIALS, a SUite of Nonlinear and DIfferential/ALgebraic
  Equation Solvers.}
  Solving nonlinear algebraic and differential equations is both a
  common task and one that often requires sophisticated globalization
  algorithms for efficiency and reliability. SUNDIALS \cite{sundials}
  provides these in a widely used format, both sequentially and in parallel.

  \dealii now has interfaces to SUNDIALS's ARKode, IDA, and KINSOL sub-packages.
  ARKode is a solver library that provides adaptive-step time
  integration. IDA is a package for the solution of differential-algebraic
  equations systems in the form $F(t,y,y')=0$. KINSOL is solver for nonlinear
  algebraic systems.

%%%%%%%%%%%%%%%%%%%%%%%%%%%%%%%%%%%%%%%%%%%%%%%%%%%%%%%%%%%%%%%%%%%%%%%%%%%%%%%%
  \subsection{Use of C++11}
  \label{sec:cxx11}
\dealii first offered support for a subset of C++11 features in
version 6.2, released in 2009. The current release is the first to
\emph{require} a C++11 compiler.

Many parts of the code base have been rewritten to both support and
use the new features of C++11. In particular, \dealii now makes
extensive use of move semantics as well as range-based \texttt{for}
loops with \texttt{auto} type deduction of iterator variables. We have
also largely replaced \texttt{push\_back()} by
\texttt{emplace\_back()} when adding elements to collections more
efficiently.

Finally, we have changed the entire code base to avoid using raw
pointers and instead use \texttt{std::unique\_ptr} and
\texttt{std::shared\_ptr} where possible to make memory management
more reliable. These changes include some minor incompatibilities: all
\texttt{clone()} functions (such as \texttt{FiniteElement::clone()} and
\texttt{Mapping::clone()}) now return \texttt{std::unique\_ptr}s instead of
C-style raw pointers. Indeed, nearly all interfaces throughout the library that return a pointer
now return either a \texttt{std::shared\_ptr} or a \texttt{std::unique\_ptr},
thereby clarifying object ownership responsibilities and avoiding memory leaks.


%%%%%%%%%%%%%%%%%%%%%%%%%%%%%%%%%%%%%%%%%%%%%%%%%%%%%%%%%%%%%%%%%%%%%%%%%%%%%%%%
\subsection{Support for GPU computations}

\marginpar{Bruno?}
...



%%%%%%%%%%%%%%%%%%%%%%%%%%%%%%%%%%%%%%%%%%%%%%%%%%%%%%%%%%%%%%%%%%%%%%%%%%%%%%%%
\subsection{MeshWorker::mesh\_loop}

\marginpar{do we want this? LH}

%%%%%%%%%%%%%%%%%%%%%%%%%%%%%%%%%%%%%%%%%%%%%%%%%%%%%%%%%%%%%%%%%%%%%%%%%%%%%%%%
\subsection{Extended matrix-free capabilities}

The matrix-free infrastructure in \dealii{} was significantly overhauled for
the current release. The major new contribution is the support of face
integrals through a new class \texttt{FEFaceEvaluation}. The new class has a
similar interface as the existing \texttt{FEEvaluation} class, and applies SIMD
vectorization over several faces in analogy to the intra-cell vectorization in
FEEvaluation. Discontinuous Galerkin operators are implemented defining two
face functions, one for interior and one for boundary faces, in addition to
the cell function. These kernels for the matrix-free operator evaluation are
now collected in the new function \texttt{MatrixFree::loop}. The data
structures have been particularly tuned for typical discontinuous Galerkin
setups involving operators with first and second spatial derivatives. Both
data access and computations have been thoroughly optimized and compared to
the performance boundaries of the hardware.

To give an example of the algorithmic improvements, the computation of the
values and gradients on all quadrature points for cell integrals has been
significantly improved, yielding 10--20\% better performance for cases where
the kernels are compute bound. For the example of the reference cell gradient
of a solution field $\mathbf u$ in three space dimensions, the new release
applies the following change:
\begin{equation*}
  \text{previous: }
  \begin{bmatrix}
    D_{1} \otimes S_{2} \otimes S_{3}\\
    S_{1} \otimes D_{2} \otimes S_{3}\\
    S_{1} \otimes S_{2} \otimes D_{3}
  \end{bmatrix}
  \mathbf u
  \quad
  \leadsto
  \quad
  \text{new: }
  \begin{bmatrix}
    D_{1}^{\mathrm{co}} \otimes I_{2} \otimes I_{3}\\
    I_{1} \otimes D_{2}^{\mathrm{co}} \otimes I_{3}\\
    I_{1} \otimes I_{2} \otimes D_{3}^{\mathrm{co}}
  \end{bmatrix}
  \begin{bmatrix}
    S_{1} \otimes S_{2} \otimes S_{3}
  \end{bmatrix}
  \mathbf u.
\end{equation*}
The matrices $S_i$ contain the values of the one-dimensional shape functions
in one-dimensional quadrature points and $D_i$ their derivatives. When applied
with the usual sum factorization implementation described, for
example, in
\cite{KronbichlerKormann2012}, the old kernels amounted to 9 partial
summations -- or rather 8 in the previous implementation of \dealii{} because
the application of $S_1$ for the $y$ and $z$ components of the gradient can be
merged. The new code performs a basis transformation to a related basis with
derivative matrix $D_i = D_i^{\mathrm{co}} S_i$, which is the basis of
Lagrange polynomials in the points of the quadrature. This change reduces the
number of partial sums to only 6 for the gradient, as the action of the unit
matrices $I_i$ needs not be implemented. In isolation, this spectral
element-like evaluation was previously available in \dealii{} for collocation
between nodal points and quadrature, but not used for general bases. A more
detailed description of the matrix-free modules is given in the preprint
\cite{KronbichlerKormann2017}.


%%%%%%%%%%%%%%%%%%%%%%%%%%%%%%%%%%%%%%%%%%%%%%%%%%%%%%%%%%%%%%%%%%%%%%%%%%%%%%%%
\subsection{Tutorial and code gallery programs}

This release does not contain any new tutorial programs, though
several have been updated extensively for the changes to the manifold
handling as well as to adjust for current functionality and coding styles
compared to what that was available when the programs were first written.

\dealii has a separate ``code gallery'' that
consists of programs shared by users as examples of what can be
done with \dealii. While not part of the release process, it is nonetheless
worth mentioning that the set of new programs since the last release covers
the following topics:
\begin{itemize}
\item The multipoint flux mixed finite element method (MFMFE) applied
  to the Darcy problem of porous media flow;
\item A linearized active skeletal muscle model with application to
  the simulation concentric contraction of the human biceps brachii;
\item A parallel implementation of the Local Discontinuous Galerkin
  (LDG) method applied to the Poisson equation.
\end{itemize}
With these additions, the code gallery now contains 10 different applications.


%%%%%%%%%%%%%%%%%%%%%%%%%%%%%%%%%%%%%%%%%%%%%%%%%%%%%%%%%%%%%%%%%%%%%%%%%%%%%%%%
\subsection{Incompatible changes}

The 9.0 release includes
\href{https://www.dealii.org/developer/doxygen/deal.II/changes_between_8_5_and_9_0.html}
     {around 75 incompatible changes}; see \cite{changes90}. The majority of these changes
should not be visible to typical user codes; some remove previously
deprecated classes and functions, and the majority change internal
interfaces that are not usually used in external
applications. However, some are, such as changes to the interplay between meshes and
manifolds, as well as the requirement to use a
C++11 compiler (see Sections~\ref{sec:manifolds} and
~\ref{sec:cxx11}). In addition, the following
incompatible changes are worth mentioning:
\begin{itemize}
\item The \texttt{BlockDiagonalMatrix},
  \texttt{InverseMatrixRichardson},
  \texttt{IterativeInverse},
  \texttt{ProductMatrix},
  \texttt{ProductSparseMatrix},
  \texttt{TransposeMatrix},
  \texttt{ScaledMatrix},
  \texttt{SchurMatrix},
  \texttt{ShiftedMatrix}, and
  \texttt{ShiftedMatrixGeneralized} classes have been removed. They are now
  generalized through the \texttt{LinearOperator} concept. Several
  other, similar classes have been deprecated.

  \item The default partitioner for the
    \texttt{parallel::shared::Triangulation} is now the Trilinos
    package Zoltan. This functionality was previously provided by the
    METIS partitioner, but the METIS package has not been actively maintained
    for a long time, and moreover yields subdivisions that depend on
    system details such as the random number generator and sorting
    facilities provided by the operating system; consequently, the
    partition is not consistent across platforms.

  \item  The class \texttt{FE\_DGQHermite}  now uses a more stable,
    ``Hermite-like'' polynomial basis. The change is highly
    beneficial because it significantly improves the accuracy (in
    terms of roundoff) for this basis and also reduces iteration
    counts for some iterative solvers with simple preconditioners.

  \item Many functions that previously returned a raw, C-style pointer
    now return a \texttt{std::unique\_ptr} and
    \texttt{std::shared\_ptr} where possible to make memory management
    more reliable.
\end{itemize}



%%%%%%%%%%%%%%%%%%%%%%%%%%%%%%%%%%%%%%%%%%%%%%%%%%%%%%%%%%%%%%%%%%%%%%%%%%%%%%%%
%%%%%%%%%%%%%%%%%%%%%%%%%%%%%%%%%%%%%%%%%%%%%%%%%%%%%%%%%%%%%%%%%%%%%%%%%%%%%%%%
%%%%%%%%%%%%%%%%%%%%%%%%%%%%%%%%%%%%%%%%%%%%%%%%%%%%%%%%%%%%%%%%%%%%%%%%%%%%%%%%
\section{How to cite \dealii}\label{sec:cite}

In order to justify the work the developers of \dealii put into this
software, we ask that papers using the library reference one of the
\dealii papers. This helps us justify the effort we put into it.

There are various ways to reference \dealii. To acknowledge the use of
the current version of the library, \textbf{please reference the present
document}. For up to date information and bibtex snippets for this document
see:
\begin{center}
 \url{https://www.dealii.org/publications.html}
\end{center}

The original \texttt{\dealii} paper containing an overview of its
architecture is \cite{BangerthHartmannKanschat2007}. If you rely on
specific features of the library, please consider citing any of the
following:
\begin{itemize}
 \item For geometric multigrid: \cite{Kanschat2004,JanssenKanschat2011};
 \item For distributed parallel computing: \cite{BangerthBursteddeHeisterKronbichler11};
 \item For $hp$ adaptivity: \cite{BangerthKayserHerold2007};
  \item For partition-of-unity (PUM) and enrichment methods of the
    finite element space: \cite{Davydov2016};
 \item For matrix-free and fast assembly techniques:
   \cite{KronbichlerKormann2012};
 \item For computations on lower-dimensional manifolds:
   \cite{DeSimoneHeltaiManigrasso2009};
 \item For integration with CAD files and tools:
   \cite{HeltaiMola2015};
 \item For \texttt{LinearOperator} and \texttt{PackagedOperation} facilities:
   \cite{MaierBardelloniHeltai-2016-a,MaierBardelloniHeltai-2016-b}.
 \item For uses of the \texttt{WorkStream} interface:
   \cite{TKB16}.
\end{itemize}

\dealii can interface with many other libraries:
\begin{multicols}{3}
\begin{itemize}
\item ADOL-C \cite{Griewank1996a,adol-c}
\item ARPACK \cite{arpack}
\item Assimp \cite{assimp}
\item BLAS and LAPACK \cite{lapack}
\item Gmsh \cite{geuzaine2009gmsh}
\item GSL \cite{gsl2016}
\item HDF5 \cite{hdf5}
\item METIS \cite{karypis1998fast}
\item MUMPS \cite{ADE00,MUMPS:1,MUMPS:2,mumps-web-page}
\item muparser \cite{muparser-web-page}
\item nanoflann \cite{nanoflann}
\item NetCDF \cite{rew1990netcdf}
\item OpenCASCADE \cite{opencascade-web-page}
\item p4est \cite{p4est}
\item PETSc \cite{petsc-user-ref,petsc-web-page}
\item ROL \cite{ridzal2014rapid}
\item ScaLAPACK \cite{slug}
\item SLEPc \cite{Hernandez:2005:SSF}
\item SUNDIALS \cite{sundials}
\item TBB \cite{Rei07}
\item Trilinos \cite{trilinos,trilinos-web-page}
\item UMFPACK \cite{umfpack}
\end{itemize}
\end{multicols}
Please consider citing the appropriate references if you use interfaces to these
libraries.

Older releases of \dealii can be cited as
\cite{dealII80,dealII81,dealII82,dealII83,dealII84,dealII85}.

\nocite{BangerthKanschat1999}

\section{Acknowledgments}

\dealii is a world-wide project with dozens of contributors around the
globe. Other than the authors of this paper, the following people contributed code to
this release:\\
%
% get this from the changes/*/* files using the command listed in the
% release-tasks paper and remove the authors of this paper
%
% Updated 2018-04-19
Julian Andrej,
Rajat Arora,
Lucas Campos,
Praveen Chandrashekar,
Jie Cheng,
Emma Cinatl,
Conrad Clevenger,
Ester Comellas,
Sambit Das,
Giovanni Di Ilio,
Nivesh Dommaraju,
Marc Fehling,
Niklas Fehn,
Menno Fraters,
Anian Fuchs,
Daniel Garcia-Sanchez,
Nicola Giuliani,
Anne Glerum,
Christoph Goering,
Alexander Grayver,
Samuel Imfeld,
Daniel Jodlbauer,
Guido Kanschat,
Vishal Kenchan,
Andreas Kergassner,
Eldar Khattatov,
Ingo Kligge,
Uwe Köcher,
Joachim Kopp,
Ross Kynch,
Konstantin Ladutenko,
Tulio Ligneul,
Karl Ljungkvist,
Santiago Ospina,
Alexey Ozeritsky,
Dirk Peschka,
Simon Puchert,
E. G. Puckett,
Lei Qiao,
Ce Qin,
Jonathan Robey,
Alberto Sartori,
Daniel Shapero,
Ben Shields,
Simon Sticko,
Oliver Sutton,
Zhuoran Wang,
Xiaoyu Wei,
Michał Wichrowski,
Julius Witte,
Feimi Yu,
Weixiong Zheng.

Their contributions are much appreciated!


\bigskip

\dealii and its developers are financially supported through a
variety of funding sources:
\marginpar{all: update your funding sources}

D.~Arndt, K.~Kormann and M.~Kronbichler were partially supported by the German
Research Foundation (DFG) under the project ``High-order discontinuous
Galerkin for the exa-scale'' (\mbox{ExaDG}) within the priority program ``Software
for Exascale Computing'' (SPPEXA).

W.~Bangerth and R.~Gassm\"{o}ller were partially
supported by the National Science Foundation under award OCI-1148116
as part of the Software Infrastructure for Sustained Innovation (SI2)
program; and by the Computational Infrastructure in Geodynamics initiative
(CIG), through the National Science Foundation under Awards
No.~EAR-0949446 and EAR-1550901 and The University of California -- Davis.

V.~Boddu was supported by the German Research Foundation (DFG) under the
research group project FOR 1509.

B.~Brands was partially supported by the Indo-German exchange programm
``Multiscale Modeling, Simulation and Optimization for Energy, Advanced
Materials and Manufacturing'' (MMSO) funded by DAAD (Germany) and UGC
(India).

D.~Davydov was supported by the German Research Foundation (DFG), grant DA
1664/2-1.

T.~Heister was partially supported by the Computational Infrastructure in
Geodynamics initiative (CIG), through the National Science Foundation (NSF)
under Award No. EAR-0949446 and EAR-1550901 and The University of
California -- Davis, and NSF Grant DMS-1522191.

M.~Maier was partially supported by ARO MURI Award No. W911NF-14-0247.

J-P.~Pelteret was supported by the European Research Council (ERC) through
the Advanced Grant 289049 MOCOPOLY.

B.~Turcksin: This material is based upon work supported by the U.S.
Department of Energy, Office of Science, under contract number
DE-AC05-00OR22725.

D.~Wells was supported by the National Science Foundation (NSF) through Grant
DMS-1344962.

The Interdisciplinary Center for Scientific Computing (IWR) at Heidelberg
University has provided hosting services for the \dealii web page.


\bibliography{paper}{}
\bibliographystyle{abbrv}

\end{document}
